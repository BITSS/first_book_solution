% Options for packages loaded elsewhere
\PassOptionsToPackage{unicode}{hyperref}
\PassOptionsToPackage{hyphens}{url}
%
\documentclass[
]{book}
\usepackage{amsmath,amssymb}
\usepackage{lmodern}
\usepackage{ifxetex,ifluatex}
\ifnum 0\ifxetex 1\fi\ifluatex 1\fi=0 % if pdftex
  \usepackage[T1]{fontenc}
  \usepackage[utf8]{inputenc}
  \usepackage{textcomp} % provide euro and other symbols
\else % if luatex or xetex
  \usepackage{unicode-math}
  \defaultfontfeatures{Scale=MatchLowercase}
  \defaultfontfeatures[\rmfamily]{Ligatures=TeX,Scale=1}
\fi
% Use upquote if available, for straight quotes in verbatim environments
\IfFileExists{upquote.sty}{\usepackage{upquote}}{}
\IfFileExists{microtype.sty}{% use microtype if available
  \usepackage[]{microtype}
  \UseMicrotypeSet[protrusion]{basicmath} % disable protrusion for tt fonts
}{}
\makeatletter
\@ifundefined{KOMAClassName}{% if non-KOMA class
  \IfFileExists{parskip.sty}{%
    \usepackage{parskip}
  }{% else
    \setlength{\parindent}{0pt}
    \setlength{\parskip}{6pt plus 2pt minus 1pt}}
}{% if KOMA class
  \KOMAoptions{parskip=half}}
\makeatother
\usepackage{xcolor}
\IfFileExists{xurl.sty}{\usepackage{xurl}}{} % add URL line breaks if available
\IfFileExists{bookmark.sty}{\usepackage{bookmark}}{\usepackage{hyperref}}
\hypersetup{
  pdftitle={First Book Using DD},
  pdfauthor={Your Name},
  hidelinks,
  pdfcreator={LaTeX via pandoc}}
\urlstyle{same} % disable monospaced font for URLs
\usepackage{color}
\usepackage{fancyvrb}
\newcommand{\VerbBar}{|}
\newcommand{\VERB}{\Verb[commandchars=\\\{\}]}
\DefineVerbatimEnvironment{Highlighting}{Verbatim}{commandchars=\\\{\}}
% Add ',fontsize=\small' for more characters per line
\usepackage{framed}
\definecolor{shadecolor}{RGB}{248,248,248}
\newenvironment{Shaded}{\begin{snugshade}}{\end{snugshade}}
\newcommand{\AlertTok}[1]{\textcolor[rgb]{0.94,0.16,0.16}{#1}}
\newcommand{\AnnotationTok}[1]{\textcolor[rgb]{0.56,0.35,0.01}{\textbf{\textit{#1}}}}
\newcommand{\AttributeTok}[1]{\textcolor[rgb]{0.77,0.63,0.00}{#1}}
\newcommand{\BaseNTok}[1]{\textcolor[rgb]{0.00,0.00,0.81}{#1}}
\newcommand{\BuiltInTok}[1]{#1}
\newcommand{\CharTok}[1]{\textcolor[rgb]{0.31,0.60,0.02}{#1}}
\newcommand{\CommentTok}[1]{\textcolor[rgb]{0.56,0.35,0.01}{\textit{#1}}}
\newcommand{\CommentVarTok}[1]{\textcolor[rgb]{0.56,0.35,0.01}{\textbf{\textit{#1}}}}
\newcommand{\ConstantTok}[1]{\textcolor[rgb]{0.00,0.00,0.00}{#1}}
\newcommand{\ControlFlowTok}[1]{\textcolor[rgb]{0.13,0.29,0.53}{\textbf{#1}}}
\newcommand{\DataTypeTok}[1]{\textcolor[rgb]{0.13,0.29,0.53}{#1}}
\newcommand{\DecValTok}[1]{\textcolor[rgb]{0.00,0.00,0.81}{#1}}
\newcommand{\DocumentationTok}[1]{\textcolor[rgb]{0.56,0.35,0.01}{\textbf{\textit{#1}}}}
\newcommand{\ErrorTok}[1]{\textcolor[rgb]{0.64,0.00,0.00}{\textbf{#1}}}
\newcommand{\ExtensionTok}[1]{#1}
\newcommand{\FloatTok}[1]{\textcolor[rgb]{0.00,0.00,0.81}{#1}}
\newcommand{\FunctionTok}[1]{\textcolor[rgb]{0.00,0.00,0.00}{#1}}
\newcommand{\ImportTok}[1]{#1}
\newcommand{\InformationTok}[1]{\textcolor[rgb]{0.56,0.35,0.01}{\textbf{\textit{#1}}}}
\newcommand{\KeywordTok}[1]{\textcolor[rgb]{0.13,0.29,0.53}{\textbf{#1}}}
\newcommand{\NormalTok}[1]{#1}
\newcommand{\OperatorTok}[1]{\textcolor[rgb]{0.81,0.36,0.00}{\textbf{#1}}}
\newcommand{\OtherTok}[1]{\textcolor[rgb]{0.56,0.35,0.01}{#1}}
\newcommand{\PreprocessorTok}[1]{\textcolor[rgb]{0.56,0.35,0.01}{\textit{#1}}}
\newcommand{\RegionMarkerTok}[1]{#1}
\newcommand{\SpecialCharTok}[1]{\textcolor[rgb]{0.00,0.00,0.00}{#1}}
\newcommand{\SpecialStringTok}[1]{\textcolor[rgb]{0.31,0.60,0.02}{#1}}
\newcommand{\StringTok}[1]{\textcolor[rgb]{0.31,0.60,0.02}{#1}}
\newcommand{\VariableTok}[1]{\textcolor[rgb]{0.00,0.00,0.00}{#1}}
\newcommand{\VerbatimStringTok}[1]{\textcolor[rgb]{0.31,0.60,0.02}{#1}}
\newcommand{\WarningTok}[1]{\textcolor[rgb]{0.56,0.35,0.01}{\textbf{\textit{#1}}}}
\usepackage{longtable,booktabs,array}
\usepackage{calc} % for calculating minipage widths
% Correct order of tables after \paragraph or \subparagraph
\usepackage{etoolbox}
\makeatletter
\patchcmd\longtable{\par}{\if@noskipsec\mbox{}\fi\par}{}{}
\makeatother
% Allow footnotes in longtable head/foot
\IfFileExists{footnotehyper.sty}{\usepackage{footnotehyper}}{\usepackage{footnote}}
\makesavenoteenv{longtable}
\usepackage{graphicx}
\makeatletter
\def\maxwidth{\ifdim\Gin@nat@width>\linewidth\linewidth\else\Gin@nat@width\fi}
\def\maxheight{\ifdim\Gin@nat@height>\textheight\textheight\else\Gin@nat@height\fi}
\makeatother
% Scale images if necessary, so that they will not overflow the page
% margins by default, and it is still possible to overwrite the defaults
% using explicit options in \includegraphics[width, height, ...]{}
\setkeys{Gin}{width=\maxwidth,height=\maxheight,keepaspectratio}
% Set default figure placement to htbp
\makeatletter
\def\fps@figure{htbp}
\makeatother
\setlength{\emergencystretch}{3em} % prevent overfull lines
\providecommand{\tightlist}{%
  \setlength{\itemsep}{0pt}\setlength{\parskip}{0pt}}
\setcounter{secnumdepth}{5}
\usepackage{booktabs}
\ifluatex
  \usepackage{selnolig}  % disable illegal ligatures
\fi
\usepackage[]{natbib}
\bibliographystyle{apalike}

\title{First Book Using DD}
\author{Your Name}
\date{2021-09-02}

\begin{document}
\maketitle

{
\setcounter{tocdepth}{1}
\tableofcontents
}
\hypertarget{hands-on-excercise-the-birthday-problem}{%
\chapter{Hands-on excercise: the birthday problem!}\label{hands-on-excercise-the-birthday-problem}}

As an illustration lets write a report using the participants in this workshop to illustrate the famous \href{https://en.wikipedia.org/wiki/Birthday_problem}{birthday problem}.

\begin{quote}
What is the probability that at least two people this room share the same birthday?
\end{quote}

\begin{quote}
There are 21 in this room.
\end{quote}

\begin{quote}
Is it something like \(\frac{21}{365} =\) 0.058?
\end{quote}

\hypertarget{the-birthday-problem-the-math}{%
\chapter{The birthday problem: the math}\label{the-birthday-problem-the-math}}

Actually the math says otherwise. Define p(n) as the probability that at least one pair has the same birthday, then the \(1 - p(n)\) is the probability that all are born in a different day. Which we can compute as:

--

\begin{align} 
 1 -  p(n) &= 1 \times \left(1-\frac{1}{365}\right) \times \left(1-\frac{2}{365}\right) \times \cdots \times \left(1-\frac{n-1}{365}\right) \nonumber  \newline
 &= \frac{ 365 \times 364 \times \cdots \times (365-n+1) }{ 365^n } \nonumber \newline
 &= \frac{ 365! }{ 365^n (365-n)!} = \frac{n!\cdot\binom{365}{n}}{365^n}\newline
p(n= 21) &= 0.444  \nonumber
\end{align}

\hypertarget{code-for-the-math}{%
\section{Code for the math}\label{code-for-the-math}}

(\texttt{/dynamicdocs/first\_dd\_solution.Rmd})

Copy and paste lines below into your \texttt{first\_dd.Rmd}

\begin{Shaded}
\begin{Highlighting}[]
\KeywordTok{\textbackslash{}begin}\NormalTok{\{}\ExtensionTok{align}\NormalTok{\}}\SpecialStringTok{ }
\SpecialStringTok{ 1 {-}  p(n) \&= 1 }\SpecialCharTok{\textbackslash{}times}\SpecialStringTok{ }\SpecialCharTok{\textbackslash{}left}\SpecialStringTok{(1{-}}\SpecialCharTok{\textbackslash{}frac}\SpecialStringTok{\{1\}\{365\}}\SpecialCharTok{\textbackslash{}right}\SpecialStringTok{) }\SpecialCharTok{\textbackslash{}times}
\SpecialStringTok{              }\SpecialCharTok{\textbackslash{}left}\SpecialStringTok{(1{-}}\SpecialCharTok{\textbackslash{}frac}\SpecialStringTok{\{2\}\{365\}}\SpecialCharTok{\textbackslash{}right}\SpecialStringTok{) }\SpecialCharTok{\textbackslash{}times}\SpecialStringTok{ }\SpecialCharTok{\textbackslash{}cdots}\SpecialStringTok{ }\SpecialCharTok{\textbackslash{}times}
\SpecialStringTok{              }\SpecialCharTok{\textbackslash{}left}\SpecialStringTok{(1{-}}\SpecialCharTok{\textbackslash{}frac}\SpecialStringTok{\{n{-}1\}\{365\}}\SpecialCharTok{\textbackslash{}right}\SpecialStringTok{) }\SpecialCharTok{\textbackslash{}nonumber}\SpecialStringTok{  }\SpecialCharTok{\textbackslash{}newline}
\SpecialStringTok{           \&= }\SpecialCharTok{\textbackslash{}frac}\SpecialStringTok{\{365 }\SpecialCharTok{\textbackslash{}times}\SpecialStringTok{ 364 }\SpecialCharTok{\textbackslash{}times}\SpecialStringTok{ }\SpecialCharTok{\textbackslash{}cdots}\SpecialStringTok{ }\SpecialCharTok{\textbackslash{}times}\SpecialStringTok{ (365{-}n+1) \}\{365\^{}n\} }\SpecialCharTok{\textbackslash{}nonumber}\SpecialStringTok{ }\SpecialCharTok{\textbackslash{}newline}
\SpecialStringTok{           \&= }\SpecialCharTok{\textbackslash{}frac}\SpecialStringTok{\{ 365! \}\{ 365\^{}n (365{-}n)!\} = }\SpecialCharTok{\textbackslash{}frac}\SpecialStringTok{\{n!}\SpecialCharTok{\textbackslash{}cdot\textbackslash{}binom}\SpecialStringTok{\{365\}\{n\}\}\{365\^{}n\}}\SpecialCharTok{\textbackslash{}newline}
\SpecialCharTok{\textbackslash{}p}\SpecialStringTok{(n= \textasciigrave{}r n.pers\textasciigrave{}) \&= \textasciigrave{}r  round(1 {-} factorial(n.pers) * }
\SpecialStringTok{                          choose(365,n.pers)/ 365\^{}n.pers, 3)\textasciigrave{}  }\SpecialCharTok{\textbackslash{}nonumber}
\KeywordTok{\textbackslash{}end}\NormalTok{\{}\ExtensionTok{align}\NormalTok{\}}
\end{Highlighting}
\end{Shaded}

\hypertarget{dont-like-math-lets-run-a-simple-simulation}{%
\chapter{Don't like math? Let's run a simple simulation!}\label{dont-like-math-lets-run-a-simple-simulation}}

1 - Simulate \ensuremath{10^{4}} rooms with \(n = 21\) random birthdays, and store the results in matrix where each row represents a room.

2 - For each room (row) compute the number of unique birthdays.

3 - Compute the average number of times a room has 21 unique birthdays, across \ensuremath{10^{4}} simulations, and report the complement.

\hypertarget{code-for-the-simulation}{%
\section{Code for the simulation}\label{code-for-the-simulation}}

(\texttt{/dynamicdocs/first\_dd\_solution.Rmd})

\begin{Shaded}
\begin{Highlighting}[]
\NormalTok{birthday.prob }\OtherTok{=} \ControlFlowTok{function}\NormalTok{(n.pers\_var, n.sims\_var) \{}
  \CommentTok{\# simulate birthdays}
\NormalTok{  birthdays }\OtherTok{=} \FunctionTok{matrix}\NormalTok{(}\FunctionTok{round}\NormalTok{(}\FunctionTok{runif}\NormalTok{(}\AttributeTok{n =}\NormalTok{ n.pers\_var }\SpecialCharTok{*}\NormalTok{ n.sims\_var, }\AttributeTok{min =} \DecValTok{1}\NormalTok{, }\AttributeTok{max =} \DecValTok{365}\NormalTok{) ),}
                     \AttributeTok{nrow =}\NormalTok{ n.sims\_var, }\AttributeTok{ncol =}\NormalTok{ n.pers\_var)}
  \CommentTok{\# for each room (row) get unique birthdays}
\NormalTok{  unique.birthdays }\OtherTok{=} \FunctionTok{apply}\NormalTok{(birthdays, }\DecValTok{1}\NormalTok{,}
                           \ControlFlowTok{function}\NormalTok{(x)  }\FunctionTok{length}\NormalTok{( }\FunctionTok{unique}\NormalTok{(x) ) )}
  \CommentTok{\# Indicator with 1 if all are unique birthdays}
\NormalTok{  all.different }\OtherTok{=} \DecValTok{1} \SpecialCharTok{*}\NormalTok{ (unique.birthdays}\SpecialCharTok{==}\NormalTok{n.pers\_var)}
  \CommentTok{\# Compute average time all have different birthdays }
\NormalTok{  result }\OtherTok{=} \DecValTok{1} \SpecialCharTok{{-}} \FunctionTok{mean}\NormalTok{(all.different)}
\FunctionTok{return}\NormalTok{(result)}
\NormalTok{\}}

\NormalTok{bp\_sim }\OtherTok{=} \FunctionTok{birthday.prob}\NormalTok{(}\AttributeTok{n.pers\_var =} \DecValTok{21}\NormalTok{, }\AttributeTok{n.sims\_var =} \DecValTok{10000}\NormalTok{)}
\FunctionTok{print}\NormalTok{(bp\_sim)}
\end{Highlighting}
\end{Shaded}

\begin{verbatim}
## [1] 0.4531
\end{verbatim}

\hypertarget{results}{%
\chapter{Results}\label{results}}

\begin{itemize}
\item
  Many people originally think of a prob \textasciitilde{} \(\frac{1}{365} \times n = 0.058\)
\item
  However the true probability is of \(p(n= 21) = 0.444\)
\item
  And the simulated probability is of \(0.4531\)
  {]}
\end{itemize}

\end{document}
